\maketitle

This problem set is due on Wednesday, March 15 at 11:59 pm. Each problem part is worth 3 points. Collaboration is encouraged. In all cases, you must write your own solutions, and and you must cite collaborators and resources used.

\begin{problem}
  Here are come exercises in using Cauchy's integral formula (and Cauchy's integral formula for higher derivatives).
  \begin{enumerate}[(a)]
    \item Compute
    \[\int_{|z|=1}\frac{e^z}z\,dz.\]
    \item Compute
    \[\int_{|z|=2}\frac{dz}{z^2+1},\]
    by decomposition of the integrand into partial fractions.
    \item Compute
    \[\int_{|z|=\rho}\frac{|dz|}{|z-a|^2}.\]
    under the condition that $|a|\neq\rho$.

    \emph{Hint}: First prove that the equations $z\overline z=\rho^2$ and $|dz|=-i\rho\frac{dz}z$ hold on $|z|=\rho$. (For the second equation, use the standard parametrization.)

    \item Using Cauchy's integral formula for higher derivatives, compute
    \[\int_{|z|=1}e^zz^{-n}\,dz.\]
  \end{enumerate}
\end{problem}

\begin{problem}
  \leavevmode\begin{enumerate}[(a)]
    \item Factor the rational number $420/69$ as a product of integral powers of primes. Then complete the following statement into a theorem: ``Every rational number can be factored uniquely (up to a factor of $\pm 1$) into\ldots.'' Prove that this theorem is true as a consequence of the fundamental theorem of arithmetic. (Look up the fundamental theorem of arithmetic if you are not sure what it is.)
    \item Factor the rational function $f(z)=\frac{x^3-1}{x^4-4x^3+6x^2-4x+1}$ as a product of integral powers of linear polynomials over $\bC$. Then complete the following statement into a theorem: ``Every nonzero rational function over $\bC$ can be factored uniquely (up to a constant) into\ldots.'' Prove that this theorem is true as a consequence of the fundamental theorem of algebra. Can you see a connection (an analogy, maybe) between part (a) and part (b)? For example, what are the ``primes'' in this setting?
    \item Do Problem 1(b) on Exam 1 on the function in part (b).
  \end{enumerate}
\end{problem}

\begin{problem}
  Show that the successive derivatives of a holomorphic function $f$ at a point $z$ cannot satisfy $|f^{(n)}(z)|>n!n^n$ for all positive integers $n$.
\end{problem}

\begin{problem}
  Prove that a function $f$ which is holomorphic in the whole complex plane and satisfies the inequality $|f(z)|<|z|^n$ for some $n$ and all sufficiently large $z$ reduces to a polynomial.
\end{problem}

\begin{problem}
  Show that a function which is analytic in the whole plane and has a nonessential singularity at $\infty$ reduces to a polynomial.
\end{problem}

\begin{problem}
  Show that the functions $e^z$, $\sin z$, and $\cos z$ have essential singularities at $\infty$.
\end{problem}

\begin{problem}
  This is a space to reflect on something about this problem set. You can mention if you found any problems particularly difficult, or particularly easy. You can also mention problems you liked, or problems that took a long time, etc. (Please write something here to get credit!)
\end{problem}

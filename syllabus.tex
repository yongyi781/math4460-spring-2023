\documentclass[11pt,oneside]{amsart}
\usepackage{geometry}
\usepackage{parskip,hyperref}

\pagestyle{empty}

\title{Course information for MATH4460 (Spring 2023)\\
    Complex Variables}

\begin{document}
\maketitle

\textbf{Instructor:} Yongyi Chen\\
\textbf{Email:} \href{mailto:yongyi.chen@bc.edu}{\texttt{yongyi.chen@bc.edu}}

\textbf{Lectures:} MWF 3:00 pm--3:50 pm in Gasson Hall 301\\
\textbf{Homework:} Weekly, due on Wednesdays at 11:59 pm.

\textbf{Office:} Maloney 532\\
\textbf{Office hours:} (tentative) Mondays 4-5 pm, Wednesdays and Fridays 1-2 pm in Maloney 532.

\section{Course information}
\subsection*{Course website}
On Canvas. There you will find homework assignments, homework solutions, and supplemental course materials.

\subsection*{Course format}
In person. I may change hours or add more office hours based on demand.

\subsection*{Textbooks}
We will be following Ahlfors, \emph{Complex Analysis}, 3rd edition.

\subsection*{Homework}
There will be weekly homework, due on Wednesdays at 11:59 pm. Because homework solutions will be posted on Canvas, late homework will not be accepted. To submit your homework, upload a single PDF file to Gradescope (accessible from within the Canvas assignment page as well).

You are encouraged to collaborate on homework with your classmates, but the work that you turn in must be your own and must be written in your own words.  Working together is good; copying somebody else’s work is plagiarism.

Writing style counts as much as having the right answer (often you will be told the answer and asked to justify it). Homework solutions must be written in complete sentences, and must be clear, concise, and readable. A correct but poorly expressed solution will not receive full credit.

Typesetting your homework using LaTeX is strongly encouraged, but not required.

\subsection*{Exams and grading}
There will be two in-class exams (50 minutes each) and a final (120 minutes). Final grades will be determined by a weighted average of homework and exam scores.  Homework counts for 20\%, each in-class exam counts for 20\%, and the final counts for 40\%.

All exams will be given in class.

\subsection*{Academic integrity}
Cheating of any kind will result in a failing grade for the course and referral to the Dean’s office for disciplinary action.  For more information on academic integrity see \url{https://www.bc.edu/integrity}.

\section{List of topics}
\begin{enumerate}
    \item Complex numbers and why we care about them
    \begin{itemize}
      \item The complex plane, representation in Cartesian and polar coordinates
      \item The Riemann sphere
    \end{itemize}
    \item Analytic (holomorphic) functions
    \begin{itemize}
      \item Complex limits, continuity, and derivative, Cauchy-Riemann equations
      \item Analytic (holomorphic) functions, first examples
      \item The complex exponential and trigonometric functions, and periodicity
      \item Multi-valued functions and the complex logarithm
    \end{itemize}
    \item Complex integration
    \begin{itemize}
      \item Line integrals
      \item Cauchy's Theorem
      \item Cauchy's integral formula
      \item Local properties of holomorphic functions: zeros and poles, maximum principle
      \item Residue theorem and applications
    \end{itemize}
    \item Extra optional topics (decided by popular demand)
    \begin{itemize}
      \item Infinite series and product expansions of holomorphic/meromorphic functions
      \item Analytic continuation
      \item Special functions (Gamma function, Riemann zeta function, elliptic functions)
      \item Möbius transformations and the upper half plane
      \item Proof of the prime number theorem
      \item Riemann surfaces
    \end{itemize}
\end{enumerate}

\end{document}
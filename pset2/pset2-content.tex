\maketitle

This problem set is due on Wednesday, February 8 at 11:59 pm. Each problem part is worth 3 points. Collaboration is encouraged. In all cases, you must write your own solutions, and and you must cite collaborators and resources used.

\begin{problem}
  For which complex numbers $a$ is $\overline z$ complex differentiable at $a$? What about $z\overline z$?
\end{problem}
\begin{solution}
Suppose $z = x + iy$. Then $\overline{z} = x - iy$, where $x,y \in \mathbb{R}$. To be holomorphic, means \[
\begin{cases}
	\frac{\partial u}{\partial x} &=\frac{\partial v}{\partial y}\\
	\frac{\partial u}{\partial y} &=-\frac{\partial v}{\partial x}
\end{cases}\]
But the first inequality is not true as the first inequality will not be satisfied ($\frac{\partial u}{\partial x} = 1, \frac{\partial v}{\partial y}= -1$), for any value of $x$, so $\bar{z}$ is nowhere differentiable.\\
For $f(z) = z\bar{z} = x^2 + y^2$. $u(x,y) = x^2 + y^2, v(x,y) = 0$. Then checking Cauchy Riemann Equation, this means \[
\begin{cases}
	\frac{\partial u}{\partial x} &=\frac{\partial v}{\partial y} = 2x\\
	\frac{\partial u}{\partial y} &=2y=-\frac{\partial v}{\partial x}=0
\end{cases}\]
The equality above will only hold, when $x = y = 0$, hence the thing is only differentiable at 0.
\end{solution}
\begin{problem}
  Prove or disprove the holomorphicity of the following functions (recall that a complex function is holomorphic if it is complex differentiable wherever it is defined):
  \begin{enumerate}[(a)]
    \item $\Re z+\Im z$
    \item $f(x+yi)=x^2+y^2i$
    \item $z^3$
    \item $z/|z|$ (note: the domain of this function is $\bC-\{0\}$)
  \end{enumerate}
\end{problem}

\begin{problem}
  Show that a holomorphic function cannot have a constant absolute value without reducing to a constant.
\end{problem}

\begin{problem}
  Use the method of the text (Section 2.1.4) to develop
  \[\frac{z^4}{z^3-1}.\]
  (\textbf{Note}: The method is \emph{not} the usual partial fractions method you learned in calculus.)\footnote{Hint: Start by doing polynomial division to take care of the pole at $\infty$ first. To check your final answer, here it is: $z+\frac 1{3(z-1)}+\frac{\omega^2}{3(z-\omega)}+\frac{\omega}{3(z-\omega^2)}$, where $\omega=e^{2\pi i/3}$. The computation \textbf{is} messy, despite the nice answer. We will revisit this problem when we talk about the residue theorem to find out why.}
\end{problem}

\begin{problem}
  Expand $(1-z)^{-m}$, $m$ a positive integer, in powers of $z$.
\end{problem}

\begin{problem}
  Find the values of $\sin i$, $\cos i$, and $\tan(1+i)$.
\end{problem}

\begin{problem}
  This is a space to reflect on something about this problem set. You can mention if you found any problems particularly difficult, or particularly easy. You can also mention problems you liked, or problems that took a long time, etc. (Please write something here to get credit!)
\end{problem}

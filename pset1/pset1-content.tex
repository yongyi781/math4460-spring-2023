\maketitle

This problem set is due on Wednesday, February 1 at 11:59 pm. Each problem part is worth 3 points. Collaboration is encouraged. In all cases, you must write your own solutions, and and you must cite collaborators and resources used.

\begin{problem}
    \leavevmode\begin{enumerate}[(a)]
      \item Find the value of $(1+2i)^3$.
      \item Find the value of $\frac 5{-3+4i}$.
      \item Find the value of $\left(\frac{2+i}{3-2i}\right)^2$.
      \item Find the value of $(1+i)^n+(1-i)^n$ where $n$ stands for a non-negative integer.
      \item Find the value of $\left(\frac{1+i}{\sqrt2}\right)^{1337}$.
    \end{enumerate}
\end{problem}
\begin{solution}
  
\end{solution}

% \begin{problem}
%   Show that
%   \[\left( \frac{-1+i\sqrt 3}{2} \right)^3=1.\]
% \end{problem}

\begin{problem}
  I showed in class that $\sqrt[3]2-\sqrt[3]4$ is a root of $x^3+6x+2$, then mentioned the other 2 roots in passing. What are they? Show that they are also roots of $x^3+6x+2$.
\end{problem}

\begin{problem}
  \leavevmode\begin{enumerate}[(a)]
    \item Factor the polynomial $x^8-1$ as much as you can over the rational numbers.
    \item Find all 8th roots of unity and express them in both Cartesian and polar form. Part (a) may be helpful here.
    \item Which of the ones you have found are primitive? Recall that a primitive 8th root of unity is an 8th root of unity which is not a $k$th root of unity for any $k<8$.
  \end{enumerate}
\end{problem}

\begin{problem}
  Suppose that $a$ and $b$ (complex numbers) are two diagonally opposite vertices of a square. What are the other two vertices of the square, in terms of $a$ and $b$?
\end{problem}

\begin{problem}
  The following problem has a nice solution using complex numbers:

  An equilateral triangle has vertices at $(0,0)$, $(a,7)$, and $(b,8)$ for some real numbers $a,b$. What is the side length of this equilateral triangle? (Recall that an equilateral triangle has all lengths equal, and all angles equal to $\pi/3$.)

  % \emph{Hint}: Let $z=a+7i$ and $w=b+8i$. From the information given in the problem, deduce that $|z|=|w|$. In fact, deduce that $w/z=e^{\pm i\pi/3}$. (Taking $+$ in the exponent means that $w$ is 60 degrees counterclockwise from $z$, and taking $-$ means vice versa.). And you know $e^{\pm i\pi/3}$ in Cartesian form!
\end{problem}

\begin{problem}
  This is a space to reflect on something about this problem set. You can mention if you found any problems particularly difficult, or particularly easy. You can also mention problems you liked, or problems that took a long time, etc. (Please write something here to get credit!)
\end{problem}
